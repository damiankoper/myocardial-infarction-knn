\section{First Section}
\subsection{A Subsection Sample}
Please note that the first paragraph of a section or subsection is
not indented. The first paragraph that follows a table, figure,
equation etc. does not need an indent, either.

Subsequent paragraphs, however, are indented.

\subsubsection{Sample Heading (Third Level)} Only two levels of
headings should be numbered. Lower level headings remain unnumbered;
they are formatted as run-in headings.

\paragraph{Sample Heading (Fourth Level)}
The contribution should contain no more than four levels of
headings. Table~\ref{tab1} gives a summary of all heading levels.

\begin{table}
    \caption{Table captions should be placed above the
        tables.}\label{tab1}
    \begin{tabular}{|l|l|l|}
        \hline
        Heading level     & Example                                          & Font size and style \\
        \hline
        Title (centered)  & {\Large\bfseries Lecture Notes}                  & 14 point, bold      \\
        1st-level heading & {\large\bfseries 1 Introduction}                 & 12 point, bold      \\
        2nd-level heading & {\bfseries 2.1 Printing Area}                    & 10 point, bold      \\
        3rd-level heading & {\bfseries Run-in Heading in Bold.} Text follows & 10 point, bold      \\
        4th-level heading & {\itshape Lowest Level Heading.} Text follows    & 10 point, italic    \\
        \hline
    \end{tabular}
\end{table}

\begin{table}
    \caption{Table from CSV.}\label{tab2}
    \begin{tabular}{|l|l|}
        \hline
        Col 1   & Col 2  \\
        \hline
        \csvreader[no head,  late after line= \\]{data/test.csv}{1=\one,2=\two}
        {\one\  & \two }
        \hline
    \end{tabular}
\end{table}

\noindent Displayed equations are centered and set on a separate
line.
\begin{equation}
    x + y = z
\end{equation}
Please try to avoid rasterized images for line-art diagrams and
schemas. Whenever possible, use vector graphics instead (see
Fig.~\ref{fig1}).

\begin{figure}
    \includegraphics[width=\textwidth]{img/test.png}
    \caption{A figure caption is always placed below the illustration.
        Please note that short captions are centered, while long ones are
        justified by the macro package automatically.} \label{fig1}
\end{figure}

\begin{theorem}
    This is a sample theorem. The run-in headasdasding is set in bold, while
    the following text appears in italics. Definitions, lemmas,
    propositions, and corollaries are styled the same way.
\end{theorem}
%
% the environments 'definition', 'lemma', 'proposition', 'corollary',
% 'remark', and 'example' are defined in the LLNCS documentclass as well.
%
\begin{proof}
    Proofs, examples, and remarks have the initial word in italics,
    while the following text appears in normal font.
\end{proof}
For citations of references, we prefer the use of square brackets
and consecutive numbers. Citations using labels or the author/year
convention are also acceptable. The following bibliography provides
a sample reference list with entries for journal
articles~\cite{abramowitz+stegun}.