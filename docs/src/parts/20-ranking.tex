\section{Wprowadzenie}
Za pomocą rankingu cech można wyróżnić najważniejsze cechy ze zbioru wszystkich cech. Jest to konieczne do przeprowadzenia procesu diagnozy. Aby wyznaczyć wspomniany ranking wykorzystano metodę SelectKBest, która znajduje się w bibliotece scikit-learn. Podana metoda przyjmuje dwa parametry: funkcję liczącą i parametr definiujący liczbę cech. Zastosowana funkcja licząca to \emph {f-classif}, która oblicza wartość ANOVA, czyli wartość analizy wariancji, dla dostarczonej próbki w celu porównania rozproszenia zmiennej zależnej w badanych grupach wyodrębnionych ze względu na wartości zmiennych niezależnych.

%tu input z tabelą z rankingiem cech

\section{Algorytm k-NN}
Algorytm k-Najblizszych Sąsiadów stanowi jedną z najbardziej prostych metod klasyfikacji. Jest łatwy w implementacji w swojej podstawowej formie oraz wykonuje dość złożone zadania klasyfikacyjne. k-NN to przykład klasyfikatora leniwego, czyli takiego, który wyciąga wnioski dopiero przy procedurze predykcji bazując na wiedzy o danych zebranej podczas procesu uczenia. W następnych etapach swojego działania omawiany algorytm wyszukuje k najbliższych wzorców ze zbioru uczącego, oblicza do nich odległość za pomocą metryki, jako predykcję zwraca tę klasę, która występuje częściej w obrębie lokalnego sąsiedztwa. Przyjmuje się, że k powinno być liczbą nieparzystą, żeby uniknąć remisu w przypadku problemów binarnych, niemniej jednak nie ma żadnej liczby, która byłaby najlepsza.

 \begin{algorithm}[!ht]
 \label{algorytm}
\SetKwInOut{Input}{Input}
\Input{$X$ = zestaw uczący\\$L$ = etykiety klas zestawu \\ $x_{q}$ = niesklasyfikowana próbka \\ $k$ = liczba sąsiadów }
\BlankLine
\SetAlgoVlined
\For{$(x', l')$ \in $X$}{
Oblicz odległość d($x'$, $x_{q}$)}

Posortuj rosnąco obliczone odległości elementów zestawu uczącego $X$ od $x_{q}$ \\
Policz wystąpienia każdej z klas w $L$ pośród najbliższych $k$ sąsiadów $x_{q}$ \\
Przydziel $x_{q}$ do najczęściej występującej klasy
\caption{K Nearest Neighbours}
\end{algorithm}

\subsection{Miary odległości}
Istotnym elementem algorytmu k-NN jest odległość, na podstawie której wyznacza się najbliższych sąsiadów. Wybrano dwie  metryki, które zostaną wykorzystane w projekcie.
Pierwsza metryka to odległość euklidesowa. Stanowi jedną na najczęściej wykorzystywanych metryk, za jej pomocą można obliczyć odległość między dwoma punktami (x, y) na płaszczyźnie.

\begin{center}
\begin{math}
 d_{e}\left( x,y\right)   = \sqrt {\sum _{i=1}^{n}  \left( x_{i}-y_{i}\right)^2 }
\end{math}
\end{center}

Drugą wybraną metryką jest manhattan, którą oblicza się stosując poniższy wzór.

\begin{center}
\begin{math}
 d_{m}\left( x,y\right)   = \sum_{i=1}^n |x_i-y_i|
\end{math}
\end{center}

