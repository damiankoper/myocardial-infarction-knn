\section{Wprowadzenie}
Cel projektu stanowi stworzenie programu do komputerowego wspomagania diagnozowania zawałów z wykorzystaniem algorytmu kNN.

\subsection{Problem medyczny jako zadania klasyfikacji}
Zadanie klasyfikacji w projekcie polega na tym, aby wspomóc rozpoznawanie stanów zwałowych wśród pacjentów na podstawie danych zgromadzonych podczas badań na ludziach, u których potwierdzono jedną z następujących diagnoz:

\begin{itemize}
    \item ból niepochodzący z serca,
    \item dusznica bolesna – dławica piersiowa,
    \item dusznica Prinzmetala – dławica naczynioskurczowa,
    \item pełnościenny zawał serca,
    \item podwsierdziowy zawał serca.
\end{itemize}
Rozkład wystąpień każdej z diagnoz przedstawia tabela \ref{tab:apriori}.
\\

W zadaniu klasyfikacji wyróżnić można pewne pojęcia, w celu lepszego jego opisu:

\begin{itemize}
    \item Klasa – pewna podprzestrzeń wartości zestawu danych, która w uczeniu nadzorowanym posiada swoją etykietę. Problem klasyfikacji jest odpowiedzią na pytania do jakiej klasy przyporządkować nowo napotkany zestaw wartości. Z punktu widzenia medycznego jest to zakwalifikowanie pacjenta jako zdrowego lub chorego z wyróżnieniem chorób na podstawie liczebności klas w~danych uczących.

    \item Cecha – właściwość, która opisuje daną klasę. W medycynie jest to między innymi płeć, wiek, samopoczucie, czy też wynik badań.
\end{itemize}

Wynik zadania klasyfikacji to przyporządkowanie każdego z pacjentów do jednej z wymienionych klas. Jakość klasyfikacji za pomocą klasyfikatora $k$ najbliższych sąsiadów została zbadana w zależności od liczby cech uwzględnionych podczas uczenia, a także zastosowanej metryki odległości.

\subsection{Opis cech}
Dane uczące to pięć plików tekstowych, przy czym każdy z nich odpowiada osobnej klasie i zawiera opis tego samego zestawu cech. Zbiór danych zawiera 5 klas, 59 cech oraz 901 rekordów. Opis poszczególnych cech z podziałem na ich charakter i możliwe do przyjęcia wartości zawiera tabela \ref{tab:cechy}.

\setlength{\tabcolsep}{5pt}
\tabulinesep=2.5pt

\begin{longtabu}{rX[l]ll}
    \caption{Opis zbioru cech danych uczących i treningowych klasyfikatora.}\label{tab:cechy}                                \\
    \toprule
    \textbf{L.p.} & \textbf{Cecha}                                        & \textbf{Charakter}   & \textbf{Wartości}         \\
    \endfirsthead
    \toprule
    \textbf{L.p.} & \textbf{Cecha}                                        & \textbf{Charakter}   & \textbf{Wartości}         \\
    \midrule
    \endhead
    \midrule
    & \multicolumn{3}{l}{\textit{Ogólne}}                                                                      \\*
    \cmidrule(r){2-4}

    1             & wiek                                                  & dyskretny            & liczby naturalne          \\
    2             & płeć                                                  & dychotomiczny        & 0 - K, 1 - M              \\
    \midrule
    & \multicolumn{3}{l}{\textit{Ból}}                                                                         \\*
    \cmidrule(r){2-4}
    3             & miejsce                                               & kategoryczny         & tabela \ref{tab:cecha_3}  \\
    4             & promieniowanie w klatce piersiowej                    & kategoryczny         & tabela \ref{tab:cecha_4}  \\
    5             & charakter                                             & kategoryczny         & tabela \ref{tab:cecha_5}  \\
    6             & początek występowania                                 & kategoryczny         & tabela \ref{tab:cecha_6}  \\
    7             & liczba godzin od rozpoczęcia                          & dyskretny            & liczby naturalne          \\
    8             & długość trwania poprzedniego                          & kategoryczny         & tabela \ref{tab:cecha_8}  \\
    \midrule
    & \multicolumn{3}{l}{\textit{Powiązane objawy}}                                                            \\*
    \cmidrule(r){2-4}
    9             & nudności                                              & dychotomiczny        & 0 - brak, 1 - obecny      \\
    10            & potliwość                                             & dychotomiczny        & 0 - brak, 1 - obecny      \\
    11            & kołatanie serca                                       & dychotomiczny        & 0 - brak, 1 - obecny      \\
    12            & duszności                                             & dychotomiczny        & 0 - brak, 1 - obecny      \\
    13            & zawroty głowy/omdlenia                                & dychotomiczny        & 0 - brak, 1 - obecny      \\
    14            & odbijanie                                             & dychotomiczny        & 0 - brak, 1 - obecny      \\
    \midrule
    & \multicolumn{3}{l}{\textit{Czynniki paliatywne}}                                                         \\*
    \cmidrule(r){2-4}
    15            & czynniki paliatywne                                   & kategoryczny         & tabela \ref{tab:cecha_15} \\
    \midrule
    & \multicolumn{3}{l}{\textit{Historia podobnego bólu}}                                                     \\*
    \cmidrule(r){2-4}
    16            & wcześniejszy, tego samego rodzaju w~klatce piersiowej & dychotomiczny        & 0 - brak, 1 - obecny      \\
    17            & konsultacja lekarska przy wcześniejszym bólu          & dychotomiczny        & 0 - brak, 1 - obecny      \\
    18            & wcześniejszy, powiązany z~sercem                      & dychotomiczny        & 0 - brak, 1 - obecny      \\
    19            & wcześniejszy, spowodowany zawałem                     & dychotomiczny        & 0 - brak, 1 - obecny      \\
    20            & wcześniejszy, spowodowany chorobą niedokrwienną serca & dychotomiczny        & 0 - brak, 1 - obecny      \\
    \midrule
    & \multicolumn{3}{l}{\textit{Historia medyczna}}                                                           \\*
    \cmidrule(r){2-4}
    21            & wcześniejszy zawał serca                              & dychotomiczny        & 0 - brak, 1 - obecny      \\
    22            & wcześniejsza choroba niedokrwienna serca              & dychotomiczny        & 0 - brak, 1 - obecny      \\
    23            & wcześniejszy nietypowy ból w~klatce piersiowej        & dychotomiczny        & 0 - brak, 1 - obecny      \\
    24            & niewydolność serca                                    & dychotomiczny        & 0 - brak, 1 - obecny      \\
    25            & choroba naczyń obwodowych                             & dychotomiczny        & 0 - brak, 1 - obecny      \\
    26            & przepuklina rozwory przełykowego                      & dychotomiczny        & 0 - brak, 1 - obecny      \\
    27            & nadciśnienie tętnicze                                 & dychotomiczny        & 0 - brak, 1 - obecny      \\
    28            & cukrzyca                                              & dychotomiczny        & 0 - brak, 1 - obecny      \\
    29            & palacz                                                & dychotomiczny        & 0 - brak, 1 - obecny      \\
    \midrule
    & \multicolumn{3}{l}{\textit{Obecne użycie leków}}                                                         \\*
    \cmidrule(r){2-4}
    30            & diuretyki                                             & dychotomiczny        & 0 - brak, 1 - obecny      \\
    31            & azotany                                               & dychotomiczny        & 0 - brak, 1 - obecny      \\
    32            & beta-blokery                                          & dychotomiczny        & 0 - brak, 1 - obecny      \\
    33            & digoksyna                                             & dychotomiczny        & 0 - brak, 1 - obecny      \\
    34            & niesteroidowe leki przeciwzapalne                     & dychotomiczny        & 0 - brak, 1 - obecny      \\
    35            & leki zobojętniające kwas żołądkowy, blokery~H2        & dychotomiczny        & 0 - brak, 1 - obecny      \\
    \midrule
    & \multicolumn{3}{l}{\textit{Badanie fizyczne}}                                                            \\*
    \cmidrule(r){2-4}
    36            & skurczowe ciśnienie tętnicze                          & dyskretny            & liczby naturalne          \\
    37            & rozkurczowe ciśnienie tętnicze                        & dyskretny            & liczby naturalne          \\
    38            & tętno                                                 & dyskretny            & liczby naturalne          \\
    39            & szybkość oddychania                                   & dyskretny            & liczby naturalne          \\
    40            & rzężenia                                              & dychotomiczny        & 0 - brak, 1 - obecny      \\
    41            & sinica                                                & dychotomiczny        & 0 - brak, 1 - obecny      \\
    42            & bladość                                               & dychotomiczny        & 0 - brak, 1 - obecny      \\
    43            & szmery skurczowe                                      & dychotomiczny        & 0 - brak, 1 - obecny      \\
    44            & szmery rozkurczowe                                    & dychotomiczny        & 0 - brak, 1 - obecny      \\
    45            & obrzęk                                                & dychotomiczny        & 0 - brak, 1 - obecny      \\
    46            & trzeci ton serca                                      & dychotomiczny        & 0 - brak, 1 - obecny      \\
    47            & czwarty ton serca                                     & dychotomiczny        & 0 - brak, 1 - obecny      \\
    48            & tkliwość ściany klatki piersiowej                     & dychotomiczny        & 0 - brak, 1 - obecny      \\
    49            & potliwość                                             & 0 - brak, 1 - obecny                             \\
    \midrule
    & \multicolumn{3}{l}{\textit{Badanie EKG}}                                                                 \\*
    \cmidrule(r){2-4}
    50            & nowy załamek Q                                        & dychotomiczny        & 0 - brak, 1 - obecny      \\
    51            & jakikolwiek załamek~Q                                 & dychotomiczny        & 0 - brak, 1 - obecny      \\
    52            & nowe uniesienie odcinka~ST                            & dychotomiczny        & 0 - brak, 1 - obecny      \\
    53            & jakiekolwiek uniesienie odcinka~ST                    & dychotomiczny        & 0 - brak, 1 - obecny      \\
    54            & nowe obniżenie odcinka~ST                             & dychotomiczny        & 0 - brak, 1 - obecny      \\
    55            & jakiekolwiek obniżenie odcinka~ST                     & dychotomiczny        & 0 - brak, 1 - obecny      \\
    56            & nowy odwrócony załamek~T                              & dychotomiczny        & 0 - brak, 1 - obecny      \\
    57            & jakikolwiek odwrócony załamek~T                       & dychotomiczny        & 0 - brak, 1 - obecny      \\
    58            & nowe zaburzenie przewodnictwa śródkomorowego          & dychotomiczny        & 0 - brak, 1 - obecny      \\
    59            & jakiekolwiek zaburzenie przewodnictwa śródkomorowego  & dychotomiczny        & 0 - brak, 1 - obecny      \\
    \bottomrule
\end{longtabu}
\begin{multicols}{2}
    \begin{wraptable}{c}{\linewidth}
        \caption{Opis wartości cechy \textit{miejsce~bólu}.}\label{tab:cecha_3}
        \begin{tabu}{rX[l]}
            \toprule
            \textbf{War.} & \textbf{Znaczenie}          \\
            \midrule

            1             & zamostkowy                  \\
            2             & lewa strona, w okol. serca  \\
            3             & prawa strona na wys. serca  \\
            4             & lewy bok klatki piersiowej  \\
            5             & prawy bok klatki piersiowej \\
            6             & brzuch                      \\
            7             & plecy                       \\
            8             & inne                        \\
            \bottomrule
        \end{tabu}
    \end{wraptable}

    \begin{wraptable}{c}{\linewidth}
        \caption{Opis wartości cechy \textit{promieniowanie bólu w klatce piersiowej}.}\label{tab:cecha_4}
        \begin{tabu}{rX[l]}
            \toprule
            \textbf{War.} & \textbf{Znaczenie} \\
            \midrule

            1             & szyja              \\
            2             & szczęka            \\
            3             & lewe ramię         \\
            4             & lewa ręka          \\
            5             & prawe ramię        \\
            6             & plecy              \\
            7             & brzuch             \\
            8             & inne               \\
            \bottomrule
        \end{tabu}
    \end{wraptable}

    \begin{wraptable}{c}{\linewidth}
        \caption{Opis wartości cechy \textit{charakter bólu}.}\label{tab:cecha_5}
        \begin{tabu}{rX[l]}
            \toprule
            \textbf{War.} & \textbf{Znaczenie} \\
            \midrule

            1             & szyja              \\
            2             & szczęka            \\
            3             & lewe ramię         \\
            4             & lewa ręka          \\
            5             & prawe ramię        \\
            6             & plecy              \\
            7             & brzuch             \\
            8             & inne               \\
            \bottomrule
        \end{tabu}
    \end{wraptable}

    \begin{wraptable}{c}{\linewidth}
        \caption{Opis wartości cechy \textit{początek występowania bólu}.}\label{tab:cecha_6}
        \begin{tabu}{rX[l]}
            \toprule
            \textbf{War.} & \textbf{Znaczenie} \\
            \midrule
            1             & podczas wysiłku    \\
            2             & w spoczynku        \\
            3             & podczas snu        \\
            \bottomrule
        \end{tabu}
    \end{wraptable}
    \begin{wraptable}{c}{\linewidth}
        \caption{Opis wartości cechy \textit{długość trwania ostatniego bólu}.}\label{tab:cecha_8}
        \begin{tabu}{rX[l]}
            \toprule
            \textbf{War.} & \textbf{Znaczenie} \\
            \midrule
            1             & poniżej 5 min      \\
            2             & 5 - 30 min         \\
            3             & 30 - 60 min        \\
            4             & 1 - 6 godz.        \\
            5             & 6 - 12 godz.       \\
            6             & powyżej 12 godz.   \\
            \bottomrule
        \end{tabu}
    \end{wraptable}
    \begin{wraptable}{c}{\linewidth}
        \caption{Opis wartości cechy \textit{czynniki paliatywne}.}\label{tab:cecha_15}
        \begin{tabu}{rX[l]}
            \toprule
            \textbf{War.} & \textbf{Znaczenie}                 \\
            \midrule
            1             & brak                               \\
            2             & nitrogliceryna w ciągu 5 min       \\
            3             & nitrogliceryna po upływie 5 min    \\
            4             & leki zobojętniające kwas żołądkowy \\
            5             & znieczulenie poza morfiną          \\
            6             & morfina                            \\
            \bottomrule
        \end{tabu}
    \end{wraptable}

\end{multicols}

\clearpage
\subsection{Algorytm selekcji cech według rankingu}
Za pomocą rankingu cech można wyróżnić najważniejsze cechy ze zbioru wszystkich cech. Jest to konieczne do przeprowadzenia procesu klasyfikacji. Aby wyznaczyć wspomniany ranking wykorzystano metodę ANOVA\cite{anova} - analizę wariancji. ANOVA stanowi popularną oraz często stosowaną analizę statystyczną służącą do porównania wpływu zmiennej na wartość analizowanej funkcji. Można podzielić analizę wariancji na 3 grupy:

\begin{itemize}
    \item jednoczynnikowa - wpływ każdego czynnika analizowany jest oddzielnie,
    \item wieloczynnikowa - wpływy czynników analizowane są razem,
    \item analiza wariancji dla czynników wewnątrzgrupowych.
\end{itemize}

Zdarza się także, że łączy się różne rodzaje: międzygrupową (jedno lub wieloczynnikową) z wewnątrzgrupową, co nazywa się mianem analizy wariancji w~schemacie mieszanym. Ideę analizy wariancji stanowi sprawdzenie, czy pewne zmienne niezależne wpływają na poziom zmiennej zależnej (testowanej).

W celu porównania wariancji dla analizowanych cech wykorzystano jednoczynnikową analizę wariancji. Porównanie wszystkich cech wraz z wartością statystyki $F$ dla metody ANOVA widoczne jest na wykresie na rysunku \ref{fig:by_score}. Działania matematyczne niezbędne do obliczenia statystyki $F$ przedstawia równanie \ref{eq:anova}.

\begin{align}\label{eq:anova}
    \begin{split}
        F              & = \frac{MSTR}{MSE}                                                         \\
        MSTR           & = \frac{1}{r-1} \sum_{i=1}^r n_i(\overline{x_i}-\overline{x})^2            \\
        MSE            & = \frac{1}{n-r} \sum_{i=1}^r \sum_{j=1}^{n_i} n_i(x_{ij}-\overline{x_i})^2 \\
        \text{gdzie:}                                                                               \\
        n_i            & - \text{liczba pomiarów $i$-klasy}                                         \\
        \overline{x_i} & - \text{średnia arytmetyczna wartości pomiarów $i$-klasy}                  \\
        \overline{x}   & - \text{średnia arytmetyczna wartości pomiarów wszystkich klas}            \\
        x_{ij}   & - \text{wartości pomiaru $j$ klasy $i$}            \\
        r              & - \text{liczba klas}
    \end{split}
\end{align}

W~dalszej analizie brano pod uwagę $k$ cech, dla których metoda analizy wariancji zwróciła najwyższy wynik, gdzie parametr $k$ był analizowany pod kątem liczby cech dających najlepsze rezultaty.


\begin{figure}[h]
    \includegraphics[width=\textwidth]{./img/by_score.png}
    \caption{Ranking cech na podstawie metody ANOVA} \label{fig:by_score}
\end{figure}


