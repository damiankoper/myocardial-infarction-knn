% This is samplepaper.tex, a sample chapter demonstrating the
% LLNCS macro package for Springer Computer Science proceedings;
% Version 2.20 of 2017/10/04
%
\documentclass[runningheads]{llncs}
%
\usepackage{graphicx}
\usepackage[T1]{fontenc}
% Used for displaying a sample figure. If possible, figure files should
% be included in EPS format.
%
% If you use the hyperref package, please uncomment the following line
% to display URLs in blue roman font according to Springer's eBook style:
% \renewcommand\UrlFont{\color{blue}\rmfamily}

\begin{document}
%
\institute{Politechnika Wrocławska\\Wydział Elektroniki\\Informatyka Techniczna\\Zastosowanie Informatyki w Medycynie - Projekt}
\title{Komputerowe wspomaganie diagnozowania zawałów z wykorzystaniem algorytmu kNN}
%
%\titlerunning{Abbreviated paper title}
% If the paper title is too long for the running head, you can set
% an abbreviated paper title here
%
\author{Karolina Działek 242 040 \and
Damian Koper 241 292}

\maketitle
%
%
\section{Wprowadzenie}
Cel projektu stanowi stworzenie programu do komputerowego wspomagania diagnozowania zawałów z wykorzystaniem algorytmu kNN. Dane wejściowe to pięć plików tekstowych, przy czym każdy z nich odpowiada osobnej klasie. Zbiór danych zawiera 5 klas, 59 cech oraz 901 rekordów. Wykorzystywanym językiem programowania jest Python, ponieważ wykorzystywaną biblioteką do uczenia maszynowego jest scikit-learn.

\section{Przedstawienie problemu medycznego jako zadania klasyfikacji}
W klasyfikacji nadzorowanej wzorce posiadają pewne składowe, które są istotne, aby prawidłowo przeprowadzić ten proces.

\begin{itemize}
\item Klasa – wzorzec jest przyporządkowany do pewnej klasy (uczenie nadzorowane). Problem klasyfikacji jest odpowiedzią na pytania do jakiej klasy przyporządkować nowo napotkany obiekt. Z punktu widzenia medycznego jest to zakwalifikowanie pacjenta jako zdrowego lub chorego.

\item Cecha – właściwość, która opisuje dany wzorzec. W medycynie jest to płeć, wiek, czy też wynik badań.
\end{itemize}

Zadanie klasyfikacji w projekcie polega na tym, aby wspomóc rozpoznawanie stanów zwałowych wśród pacjentów na podstawie danych zgromadzonych podczas badań na ludziach, u których potwierdzono jedną z następujących diagnoz:

\begin{itemize}
\item ból nie pochodzący z serca,
\item dusznica bolesna – dławica piersiowa,
\item dusznica Prinzmetala – dławica naczynioskurczowa,
\item pełnościenny zawał serca,
\item podwsierdziowy zawał serca.
\end{itemize}

Wynik zadania klasyfikacji to przyporządkowanie każdego z pacjentów do jednej z wymienionych klas. Jakość klasyfikacji zostanie zbadana za pomocą algorytmu k najbliższych sąsiadów, w zależności od liczby cech uwzględnionych podczas uczenia, a także zastosowanej metryki odległości.

\end{document}