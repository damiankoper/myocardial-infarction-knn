\clearpage
\subsection{Algorytm selekcji cech według rankingu}
Za pomocą rankingu cech można wyróżnić najważniejsze cechy ze zbioru wszystkich cech. Jest to konieczne do przeprowadzenia procesu klasyfikacji. Aby wyznaczyć wspomniany ranking wykorzystano metodę ANOVA\cite{anova} - analizę wariancji. ANOVA stanowi popularną oraz często stosowaną analizę statystyczną służącą do porównania wpływu zmiennej na wartość analizowanej funkcji. Można podzielić analizę wariancji na 3 grupy:

\begin{itemize}
    \item jednoczynnikowa - wpływ każdego czynnika analizowany jest oddzielnie,
    \item wieloczynnikowa - wpływy czynników analizowane są razem,
    \item analiza wariancji dla czynników wewnątrzgrupowych.
\end{itemize}

Zdarza się także, że łączy się różne rodzaje: międzygrupową (jedno lub wieloczynnikową) z wewnątrzgrupową, co nazywa się mianem analizy wariancji w~schemacie mieszanym. Ideę analizy wariancji stanowi sprawdzenie, czy pewne zmienne niezależne wpływają na poziom zmiennej zależnej (testowanej).

W celu porównania wariancji dla analizowanych cech wykorzystano jednoczynnikową analizę wariancji. Porównanie wszystkich cech wraz z wartością statystyki $F$ dla metody ANOVA widoczne jest na wykresie na rysunku \ref{fig:by_score}. Działania matematyczne niezbędne do obliczenia statystyki $F$ przedstawia równanie \ref{eq:anova}.

\begin{align}\label{eq:anova}
    \begin{split}
        F              & = \frac{MSTR}{MSE}                                                         \\
        MSTR           & = \frac{1}{r-1} \sum_{i=1}^r n_i(\overline{x_i}-\overline{x})^2            \\
        MSE            & = \frac{1}{n-r} \sum_{i=1}^r \sum_{j=1}^{n_i} n_i(x_{ij}-\overline{x_i})^2 \\
        \text{gdzie:}                                                                               \\
        n_i            & - \text{liczba pomiarów $i$-klasy}                                         \\
        \overline{x_i} & - \text{średnia arytmetyczna wartości pomiarów $i$-klasy}                  \\
        \overline{x}   & - \text{średnia arytmetyczna wartości pomiarów wszystkich klas}            \\
        x_{ij}   & - \text{wartości pomiaru $j$ klasy $i$}            \\
        r              & - \text{liczba klas}
    \end{split}
\end{align}

W~dalszej analizie brano pod uwagę $k$ cech, dla których metoda analizy wariancji zwróciła najwyższy wynik, gdzie parametr $k$ był analizowany pod kątem liczby cech dających najlepsze rezultaty.


\begin{figure}[h]
    \includegraphics[width=\textwidth]{./img/by_score.png}
    \caption{Ranking cech na podstawie metody ANOVA} \label{fig:by_score}
\end{figure}

\clearpage