\subsection{Algorytm selekcji cech według rankingu}
Za pomocą rankingu cech można wyróżnić najważniejsze cechy ze zbioru wszystkich cech. Jest to konieczne do przeprowadzenia procesu diagnozy. Aby wyznaczyć wspomniany ranking wykorzystano metodę SelectKBest, która znajduje się w bibliotece scikit-learn. Podana metoda przyjmuje dwa parametry: funkcję liczącą i parametr definiujący liczbę cech. Zastosowana funkcja licząca to \emph {f-classif}, która oblicza wartość ANOVA, czyli wartość analizy wariancji, dla dostarczonej próbki w celu porównania rozproszenia zmiennej zależnej w badanych grupach wyodrębnionych ze względu na wartości zmiennych niezależnych.

% Karolina tutaj opisz dokładniej SelectKBest i ANOVA

% z overleafa -> Przedstawienie wybranego algorytmu selekcji cech. Opis słowny + wzory oraz literatura.

% nie zapomnij podlinkować w bibliografii literaturki 

Porównanie wszystkich cech wraz z wartością funkcji wyniku dla metody ANOVA widoczne jest na wykresie na rysunku \ref{fig:by_score}.

\begin{figure}[h]
    \includegraphics[width=\textwidth]{./img/by_score.png}
    \caption{Ranking cech na podstawie metody ANOVA} \label{fig:by_score}
\end{figure}

