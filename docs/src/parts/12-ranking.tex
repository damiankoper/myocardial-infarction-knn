\subsection{Algorytm selekcji cech według rankingu}
Za pomocą rankingu cech można wyróżnić najważniejsze cechy ze zbioru wszystkich cech. Jest to konieczne do przeprowadzenia procesu diagnozy. Aby wyznaczyć wspomniany ranking wykorzystano metodę SelectKBest, która znajduje się w bibliotece scikit-learn. Podana metoda przyjmuje dwa parametry: funkcję liczącą i parametr definiujący liczbę cech. Zastosowana funkcja licząca to \emph {f-classif} (pobiera 2 tablice X i Y), która oblicza wartość ANOVA, czyli wartość analizy wariancji, dla dostarczonej próbki w celu porównania rozproszenia zmiennej zależnej w badanych grupach wyodrębnionych ze względu na wartości zmiennych niezależnych.

Analiza wariancji ANOVA stanowi popularną oraz często stosowaną analizę statystyczną. Można podzielić analizę wariancji na 3 grupy:

\begin{itemize}
    \item jednoczynnikowa analiza wariancji,
    \item wieloczynnikowa analiza wariancji,
    \item analiza wariancji dla czynników wewnątrzgrupowych.
\end{itemize}

Zdarza się także, że łączy się różne rodzaje: międzygrupową (jedno lub wieloczynnikową) z wewnątrzgrupową, co nazywa się mianem analizy wariancji w schemacie mieszanym. Ideę analizy wariancji stanowi sprawdzenie, czy pewne zmienne niezależne wpływają na poziom zmiennej zależnej (testowanej).

Porównanie wszystkich cech wraz z wartością funkcji wyniku dla metody ANOVA widoczne jest na wykresie na rysunku \ref{fig:by_score}.

\begin{figure}[h]
    \includegraphics[width=\textwidth]{./img/by_score.png}
    \caption{Ranking cech na podstawie metody ANOVA} \label{fig:by_score}
\end{figure}

