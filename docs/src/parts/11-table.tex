\setlength{\tabcolsep}{5pt}
\tabulinesep=2.5pt
\LTcapwidth=\textwidth

\begin{longtabu}{rX[l]ll}
    \caption{Opis zbioru cech danych uczących i treningowych klasyfikatora.}\label{tab:cechy}                                \\
    \toprule
    \textbf{L.p.} & \textbf{Cecha}                                        & \textbf{Charakter}   & \textbf{Wartości}         \\
    \endfirsthead
    \toprule
    \textbf{L.p.} & \textbf{Cecha}                                        & \textbf{Charakter}   & \textbf{Wartości}         \\
    \midrule
    \endhead
    \midrule
    & \multicolumn{3}{l}{\textit{Ogólne}}                                                                      \\*
    \cmidrule(r){2-4}

    1             & wiek                                                  & dyskretny            & liczby naturalne          \\
    2             & płeć                                                  & dychotomiczny        & 0 - K, 1 - M              \\
    \midrule
    & \multicolumn{3}{l}{\textit{Ból}}                                                                         \\*
    \cmidrule(r){2-4}
    3             & miejsce                                               & kategoryczny         & tabela \ref{tab:cecha_3}  \\
    4             & promieniowanie w klatce piersiowej                    & kategoryczny         & tabela \ref{tab:cecha_4}  \\
    5             & charakter                                             & kategoryczny         & tabela \ref{tab:cecha_5}  \\
    6             & początek występowania                                 & kategoryczny         & tabela \ref{tab:cecha_6}  \\
    7             & liczba godzin od rozpoczęcia                          & dyskretny            & liczby naturalne          \\
    8             & długość trwania poprzedniego                          & kategoryczny         & tabela \ref{tab:cecha_8}  \\
    \midrule
    & \multicolumn{3}{l}{\textit{Powiązane objawy}}                                                            \\*
    \cmidrule(r){2-4}
    9             & nudności                                              & dychotomiczny        & 0 - brak, 1 - obecny      \\
    10            & potliwość                                             & dychotomiczny        & 0 - brak, 1 - obecny      \\
    11            & kołatanie serca                                       & dychotomiczny        & 0 - brak, 1 - obecny      \\
    12            & duszności                                             & dychotomiczny        & 0 - brak, 1 - obecny      \\
    13            & zawroty głowy/omdlenia                                & dychotomiczny        & 0 - brak, 1 - obecny      \\
    14            & odbijanie                                             & dychotomiczny        & 0 - brak, 1 - obecny      \\
    \midrule
    & \multicolumn{3}{l}{\textit{Czynniki paliatywne}}                                                         \\*
    \cmidrule(r){2-4}
    15            & czynniki paliatywne                                   & kategoryczny         & tabela \ref{tab:cecha_15} \\
    \midrule \pagebreak
    & \multicolumn{3}{l}{\textit{Historia podobnego bólu}}                                                     \\
    \cmidrule(r){2-4}\nopagebreak  16            & wcześniejszy, tego samego rodzaju w~klatce piersiowej & dychotomiczny        & 0 - brak, 1 - obecny      \\
    17            & konsultacja lekarska przy wcześniejszym bólu          & dychotomiczny        & 0 - brak, 1 - obecny      \\
    18            & wcześniejszy, powiązany z~sercem                      & dychotomiczny        & 0 - brak, 1 - obecny      \\
    19            & wcześniejszy, spowodowany zawałem                     & dychotomiczny        & 0 - brak, 1 - obecny      \\
    20            & wcześniejszy, spowodowany chorobą niedokrwienną serca & dychotomiczny        & 0 - brak, 1 - obecny      \\
    \midrule
    & \multicolumn{3}{l}{\textit{Historia medyczna}}                                                           \\*
    \cmidrule(r){2-4}
    21            & wcześniejszy zawał serca                              & dychotomiczny        & 0 - brak, 1 - obecny      \\
    22            & wcześniejsza choroba niedokrwienna serca              & dychotomiczny        & 0 - brak, 1 - obecny      \\
    23            & wcześniejszy nietypowy ból w~klatce piersiowej        & dychotomiczny        & 0 - brak, 1 - obecny      \\
    24            & niewydolność serca                                    & dychotomiczny        & 0 - brak, 1 - obecny      \\
    25            & choroba naczyń obwodowych                             & dychotomiczny        & 0 - brak, 1 - obecny      \\
    26            & przepuklina rozwory przełykowego                      & dychotomiczny        & 0 - brak, 1 - obecny      \\
    27            & nadciśnienie tętnicze                                 & dychotomiczny        & 0 - brak, 1 - obecny      \\
    28            & cukrzyca                                              & dychotomiczny        & 0 - brak, 1 - obecny      \\
    29            & palacz                                                & dychotomiczny        & 0 - brak, 1 - obecny      \\
    \midrule
    & \multicolumn{3}{l}{\textit{Obecne użycie leków}}                                                         \\*
    \cmidrule(r){2-4}
    30            & diuretyki                                             & dychotomiczny        & 0 - brak, 1 - obecny      \\
    31            & azotany                                               & dychotomiczny        & 0 - brak, 1 - obecny      \\
    32            & beta-blokery                                          & dychotomiczny        & 0 - brak, 1 - obecny      \\
    33            & digoksyna                                             & dychotomiczny        & 0 - brak, 1 - obecny      \\
    34            & niesteroidowe leki przeciwzapalne                     & dychotomiczny        & 0 - brak, 1 - obecny      \\
    35            & leki zobojętniające kwas żołądkowy, blokery~H2        & dychotomiczny        & 0 - brak, 1 - obecny      \\
    \midrule
    \pagebreak
    & \multicolumn{3}{l}{\textit{Badanie fizyczne}}                                                            \\*
    \cmidrule(r){2-4}
    36            & skurczowe ciśnienie tętnicze                          & dyskretny            & liczby naturalne          \\
    37            & rozkurczowe ciśnienie tętnicze                        & dyskretny            & liczby naturalne          \\
    38            & tętno                                                 & dyskretny            & liczby naturalne          \\
    39            & szybkość oddychania                                   & dyskretny            & liczby naturalne          \\
    40            & rzężenia                                              & dychotomiczny        & 0 - brak, 1 - obecny      \\
    41            & sinica                                                & dychotomiczny        & 0 - brak, 1 - obecny      \\
    42            & bladość                                               & dychotomiczny        & 0 - brak, 1 - obecny      \\
    43            & szmery skurczowe                                      & dychotomiczny        & 0 - brak, 1 - obecny      \\
    44            & szmery rozkurczowe                                    & dychotomiczny        & 0 - brak, 1 - obecny      \\
    45            & obrzęk                                                & dychotomiczny        & 0 - brak, 1 - obecny      \\
    46            & trzeci ton serca                                      & dychotomiczny        & 0 - brak, 1 - obecny      \\
    47            & czwarty ton serca                                     & dychotomiczny        & 0 - brak, 1 - obecny      \\
    48            & tkliwość ściany klatki piersiowej                     & dychotomiczny        & 0 - brak, 1 - obecny      \\
    49            & potliwość                                             & 0 - brak, 1 - obecny                             \\
    \midrule
    & \multicolumn{3}{l}{\textit{Badanie EKG}}                                                                 \\*
    \cmidrule(r){2-4}
    50            & nowy załamek Q                                        & dychotomiczny        & 0 - brak, 1 - obecny      \\
    51            & jakikolwiek załamek~Q                                 & dychotomiczny        & 0 - brak, 1 - obecny      \\
    52            & nowe uniesienie odcinka~ST                            & dychotomiczny        & 0 - brak, 1 - obecny      \\
    53            & jakiekolwiek uniesienie odcinka~ST                    & dychotomiczny        & 0 - brak, 1 - obecny      \\
    54            & nowe obniżenie odcinka~ST                             & dychotomiczny        & 0 - brak, 1 - obecny      \\
    55            & jakiekolwiek obniżenie odcinka~ST                     & dychotomiczny        & 0 - brak, 1 - obecny      \\
    56            & nowy odwrócony załamek~T                              & dychotomiczny        & 0 - brak, 1 - obecny      \\
    57            & jakikolwiek odwrócony załamek~T                       & dychotomiczny        & 0 - brak, 1 - obecny      \\
    58            & nowe zaburzenie przewodnictwa śródkomorowego          & dychotomiczny        & 0 - brak, 1 - obecny      \\
    59            & jakiekolwiek zaburzenie przewodnictwa śródkomorowego  & dychotomiczny        & 0 - brak, 1 - obecny      \\
    \bottomrule
\end{longtabu}

\begin{table}[H]
    \caption{Opis wartości cechy \textit{miejsce~bólu}.}\label{tab:cecha_3}
    \begin{tabu}{rX[l]}
        \toprule
        \textbf{Wartość} & \textbf{Znaczenie}          \\
        \midrule

        1                & zamostkowy                  \\
        2                & lewa strona, w okol. serca  \\
        3                & prawa strona na wys. serca  \\
        4                & lewy bok klatki piersiowej  \\
        5                & prawy bok klatki piersiowej \\
        6                & brzuch                      \\
        7                & plecy                       \\
        8                & inne                        \\
        \bottomrule
    \end{tabu}
\end{table}

\begin{table}[H]
    \caption{Opis wartości cechy \textit{promieniowanie bólu w klatce piersiowej}.}\label{tab:cecha_4}
    \begin{tabu}{rX[l]}
        \toprule
        \textbf{Wartość} & \textbf{Znaczenie} \\
        \midrule

        1                & szyja              \\
        2                & szczęka            \\
        3                & lewe ramię         \\
        4                & lewa ręka          \\
        5                & prawe ramię        \\
        6                & plecy              \\
        7                & brzuch             \\
        8                & inne               \\
        \bottomrule
    \end{tabu}
\end{table}

\begin{table}[H]
    \caption{Opis wartości cechy \textit{charakter bólu}.}\label{tab:cecha_5}
    \begin{tabu}{rX[l]}
        \toprule
        \textbf{Wartość} & \textbf{Znaczenie} \\
        \midrule

        1                & szyja              \\
        2                & szczęka            \\
        3                & lewe ramię         \\
        4                & lewa ręka          \\
        5                & prawe ramię        \\
        6                & plecy              \\
        7                & brzuch             \\
        8                & inne               \\
        \bottomrule
    \end{tabu}
\end{table}

\begin{table}[H]
    \caption{Opis wartości cechy \textit{początek występowania bólu}.}\label{tab:cecha_6}
    \begin{tabu}{rX[l]}
        \toprule
        \textbf{Wartość} & \textbf{Znaczenie} \\
        \midrule
        1                & podczas wysiłku    \\
        2                & w spoczynku        \\
        3                & podczas snu        \\
        \bottomrule
    \end{tabu}
\end{table}

\begin{table}[H]
    \caption{Opis wartości cechy \textit{długość trwania ostatniego bólu}.}\label{tab:cecha_8}
    \begin{tabu}{rX[l]}
        \toprule
        \textbf{Wartość} & \textbf{Znaczenie} \\
        \midrule
        1                & poniżej 5 min      \\
        2                & 5 - 30 min         \\
        3                & 30 - 60 min        \\
        4                & 1 - 6 godz.        \\
        5                & 6 - 12 godz.       \\
        6                & powyżej 12 godz.   \\
        \bottomrule
    \end{tabu}
\end{table}

\begin{table}[H]
    \caption{Opis wartości cechy \textit{czynniki paliatywne}.}\label{tab:cecha_15}
    \begin{tabu}{rX[l]}
        \toprule
        \textbf{Wartość} & \textbf{Znaczenie}                 \\
        \midrule
        1                & brak                               \\
        2                & nitrogliceryna w ciągu 5 min       \\
        3                & nitrogliceryna po upływie 5 min    \\
        4                & leki zobojętniające kwas żołądkowy \\
        5                & znieczulenie poza morfiną          \\
        6                & morfina                            \\
        \bottomrule
    \end{tabu}
\end{table}

