% This is samplepaper.tex, a sample chapter demonstrating the
% LLNCS macro package for Springer Computer Science proceedings;
% Version 2.20 of 2017/10/04
%
\documentclass[runningheads]{llncs2e/llncs}
%
\usepackage{graphicx}
% Used for displaying a sample figure. If possible, figure files should
% be included in EPS format.
%
% If you use the hyperref package, please uncomment the following line
% to display URLs in blue roman font according to Springer's eBook style:
% \renewcommand\UrlFont{\color{blue}\rmfamily}
\usepackage{csvsimple}
\usepackage[polish]{babel}
\usepackage{booktabs}
\usepackage{tabu}
\usepackage{longtable}
\usepackage[labelsep=period]{caption}
\setlength\abovecaptionskip{0pt}
\captionsetup[table]{name=Tabela}
\usepackage[linesnumbered,ruled,vlined]{algorithm2e}
\usepackage{algpseudocode}
\usepackage{amsmath}


\usepackage[T1]{fontenc}
\usepackage{float}
\usepackage{multicol}
\usepackage{comment}
\setlength{\columnsep}{1cm}
\usepackage{wrapfig}
\usepackage{lscape}

\renewcommand*{\andname}{i}
\renewcommand*{\figurename}{Rysunek}
\renewcommand*{\keywordname}{{\bf Słowa kluczowe:}}
\renewcommand*{\abstractname}{{\bf Abstrakt:}}

\newcommand{\HRule}[1]{\rule{\linewidth}{#1}} 	% Horizontal rule
\begin{document}


\setlength{\tabcolsep}{5pt}
\tabulinesep=2.5pt
\LTcapwidth=\textwidth

\title{Komputerowe wspomaganie diagnozowania zawałów z wykorzystaniem algorytmu kNN}
\titlerunning{Komputerowe wspomaganie diagnozowania zawałów}

%
%\titlerunning{Abbreviated paper title}
% If the paper title is too long for the running head, you can set
% an abbreviated paper title here
%
\author{Karolina Działek\orcidID{242040} \and
    Damian Koper\orcidID{241292}
}
%
\authorrunning{K. Działek, D. Koper}
% First names are abbreviated in the running head.
% If there are more than two authors, 'et al.' is used.
%
\institute{Politechnika Wrocławska, Wydział Elektroniki, \\ wybrzeże Stanisława Wyspiańskiego 27, 50-370 Wrocław \\
    \email{\{242040, 241292\}@student.pwr.edu.pl}
}
%
\maketitle              % typeset the header of the contribution
%
\begin{abstract}
    Algorytm K-najbliżych sąsiadów (kNN) stanowi jedną z metod klasyfikacji. Jest prosty w implementacji w swojej podstawowej formie oraz wykonuje dość złożone zadania klasyfikacyjne. Cel niniejszego projektu to stworzenie programu do komputerowego wspomagania diagnozowania zawałów z wykorzystaniem algorytmu kNN. Do realizacji zadania wykorzystano pięć plików tekstowych jako dane wejściowe. Każdy z nich odpowiada osobnej klasie.

    \keywords{knn, myocardial, infarction}
\end{abstract}
%

\section{Wprowadzenie}
Cel projektu stanowi stworzenie programu do komputerowego wspomagania diagnozowania zawałów z wykorzystaniem algorytmu kNN.

\section{Problem medyczny jako zadania klasyfikacji}
Zadanie klasyfikacji w projekcie polega na tym, aby wspomóc rozpoznawanie stanów zwałowych wśród pacjentów na podstawie danych zgromadzonych podczas badań na ludziach, u których potwierdzono jedną z następujących diagnoz:

\begin{itemize}
    \item ból nie pochodzący z serca,
    \item dusznica bolesna – dławica piersiowa,
    \item dusznica Prinzmetala – dławica naczynioskurczowa,
    \item pełnościenny zawał serca,
    \item podwsierdziowy zawał serca.
\end{itemize}

\noindent
W zadaniu klasyfikacji wyróżnić można pewne pojęcia, w celu lepszego jego opisu:

\begin{itemize}
    \item Klasa – pewna podprzestrzeń wartości zestawu danych, która w uczeniu nadzorowanym posiada swoją etykietę. Problem klasyfikacji jest odpowiedzią na pytania do jakiej klasy przyporządkować nowo napotkany zestaw wartości. Z punktu widzenia medycznego jest to zakwalifikowanie pacjenta jako zdrowego lub chorego z wyróżnieniem chorób na podstawie liczebności klas w~danych uczących.

    \item Cecha – właściwość, która opisuje daną klasę. W medycynie jest to między innymi płeć, wiek, samopoczucie, czy też wynik badań.
\end{itemize}

Wynik zadania klasyfikacji to przyporządkowanie każdego z pacjentów do jednej z wymienionych klas. Jakość klasyfikacji za pomocą klasyfikatora $k$ najbliższych sąsiadów została zbadana w zależności od liczby cech uwzględnionych podczas uczenia, a także zastosowanej metryki odległości.

\subsection{Opis cech}
Dane uczące to pięć plików tekstowych, przy czym każdy z nich odpowiada osobnej klasie i zawiera opis tego samego zestawu cech. Zbiór danych zawiera 5 klas, 59 cech oraz 901 rekordów. Opis poszczególnych cech z podziałem na ich charakter i możliwe do przyjęcia wartości zawiera tabela \ref{tab:cechy}.

\setlength{\tabcolsep}{5pt}
\tabulinesep=2.5pt
\LTcapwidth=\textwidth

\begin{longtabu}{rX[l]ll}
    \caption{Opis zbioru cech danych uczących i treningowych klasyfikatora.}\label{tab:cechy}                                \\
    \toprule
    \textbf{L.p.} & \textbf{Cecha}                                        & \textbf{Charakter}   & \textbf{Wartości}         \\
    \endfirsthead
    \toprule
    \textbf{L.p.} & \textbf{Cecha}                                        & \textbf{Charakter}   & \textbf{Wartości}         \\
    \midrule
    \endhead
    \midrule
    & \multicolumn{3}{l}{\textit{Ogólne}}                                                                      \\*
    \cmidrule(r){2-4}

    1             & wiek                                                  & dyskretny            & liczby naturalne          \\
    2             & płeć                                                  & dychotomiczny        & 0 - K, 1 - M              \\
    \midrule
    & \multicolumn{3}{l}{\textit{Ból}}                                                                         \\*
    \cmidrule(r){2-4}
    3             & miejsce                                               & kategoryczny         & tabela \ref{tab:cecha_3}  \\
    4             & promieniowanie w klatce piersiowej                    & kategoryczny         & tabela \ref{tab:cecha_4}  \\
    5             & charakter                                             & kategoryczny         & tabela \ref{tab:cecha_5}  \\
    6             & początek występowania                                 & kategoryczny         & tabela \ref{tab:cecha_6}  \\
    7             & liczba godzin od rozpoczęcia                          & dyskretny            & liczby naturalne          \\
    8             & długość trwania poprzedniego                          & kategoryczny         & tabela \ref{tab:cecha_8}  \\
    \midrule
    & \multicolumn{3}{l}{\textit{Powiązane objawy}}                                                            \\*
    \cmidrule(r){2-4}
    9             & nudności                                              & dychotomiczny        & 0 - brak, 1 - obecny      \\
    10            & potliwość                                             & dychotomiczny        & 0 - brak, 1 - obecny      \\
    11            & kołatanie serca                                       & dychotomiczny        & 0 - brak, 1 - obecny      \\
    12            & duszności                                             & dychotomiczny        & 0 - brak, 1 - obecny      \\
    13            & zawroty głowy/omdlenia                                & dychotomiczny        & 0 - brak, 1 - obecny      \\
    14            & odbijanie                                             & dychotomiczny        & 0 - brak, 1 - obecny      \\
    \midrule
    & \multicolumn{3}{l}{\textit{Czynniki paliatywne}}                                                         \\*
    \cmidrule(r){2-4}
    15            & czynniki paliatywne                                   & kategoryczny         & tabela \ref{tab:cecha_15} \\
    \midrule \pagebreak
    & \multicolumn{3}{l}{\textit{Historia podobnego bólu}}                                                     \\
    \cmidrule(r){2-4}\nopagebreak  16            & wcześniejszy, tego samego rodzaju w~klatce piersiowej & dychotomiczny        & 0 - brak, 1 - obecny      \\
    17            & konsultacja lekarska przy wcześniejszym bólu          & dychotomiczny        & 0 - brak, 1 - obecny      \\
    18            & wcześniejszy, powiązany z~sercem                      & dychotomiczny        & 0 - brak, 1 - obecny      \\
    19            & wcześniejszy, spowodowany zawałem                     & dychotomiczny        & 0 - brak, 1 - obecny      \\
    20            & wcześniejszy, spowodowany chorobą niedokrwienną serca & dychotomiczny        & 0 - brak, 1 - obecny      \\
    \midrule
    & \multicolumn{3}{l}{\textit{Historia medyczna}}                                                           \\*
    \cmidrule(r){2-4}
    21            & wcześniejszy zawał serca                              & dychotomiczny        & 0 - brak, 1 - obecny      \\
    22            & wcześniejsza choroba niedokrwienna serca              & dychotomiczny        & 0 - brak, 1 - obecny      \\
    23            & wcześniejszy nietypowy ból w~klatce piersiowej        & dychotomiczny        & 0 - brak, 1 - obecny      \\
    24            & niewydolność serca                                    & dychotomiczny        & 0 - brak, 1 - obecny      \\
    25            & choroba naczyń obwodowych                             & dychotomiczny        & 0 - brak, 1 - obecny      \\
    26            & przepuklina rozwory przełykowego                      & dychotomiczny        & 0 - brak, 1 - obecny      \\
    27            & nadciśnienie tętnicze                                 & dychotomiczny        & 0 - brak, 1 - obecny      \\
    28            & cukrzyca                                              & dychotomiczny        & 0 - brak, 1 - obecny      \\
    29            & palacz                                                & dychotomiczny        & 0 - brak, 1 - obecny      \\
    \midrule
    & \multicolumn{3}{l}{\textit{Obecne użycie leków}}                                                         \\*
    \cmidrule(r){2-4}
    30            & diuretyki                                             & dychotomiczny        & 0 - brak, 1 - obecny      \\
    31            & azotany                                               & dychotomiczny        & 0 - brak, 1 - obecny      \\
    32            & beta-blokery                                          & dychotomiczny        & 0 - brak, 1 - obecny      \\
    33            & digoksyna                                             & dychotomiczny        & 0 - brak, 1 - obecny      \\
    34            & niesteroidowe leki przeciwzapalne                     & dychotomiczny        & 0 - brak, 1 - obecny      \\
    35            & leki zobojętniające kwas żołądkowy, blokery~H2        & dychotomiczny        & 0 - brak, 1 - obecny      \\
    \midrule
    \pagebreak
    & \multicolumn{3}{l}{\textit{Badanie fizyczne}}                                                            \\*
    \cmidrule(r){2-4}
    36            & skurczowe ciśnienie tętnicze                          & dyskretny            & liczby naturalne          \\
    37            & rozkurczowe ciśnienie tętnicze                        & dyskretny            & liczby naturalne          \\
    38            & tętno                                                 & dyskretny            & liczby naturalne          \\
    39            & szybkość oddychania                                   & dyskretny            & liczby naturalne          \\
    40            & rzężenia                                              & dychotomiczny        & 0 - brak, 1 - obecny      \\
    41            & sinica                                                & dychotomiczny        & 0 - brak, 1 - obecny      \\
    42            & bladość                                               & dychotomiczny        & 0 - brak, 1 - obecny      \\
    43            & szmery skurczowe                                      & dychotomiczny        & 0 - brak, 1 - obecny      \\
    44            & szmery rozkurczowe                                    & dychotomiczny        & 0 - brak, 1 - obecny      \\
    45            & obrzęk                                                & dychotomiczny        & 0 - brak, 1 - obecny      \\
    46            & trzeci ton serca                                      & dychotomiczny        & 0 - brak, 1 - obecny      \\
    47            & czwarty ton serca                                     & dychotomiczny        & 0 - brak, 1 - obecny      \\
    48            & tkliwość ściany klatki piersiowej                     & dychotomiczny        & 0 - brak, 1 - obecny      \\
    49            & potliwość                                             & 0 - brak, 1 - obecny                             \\
    \midrule
    & \multicolumn{3}{l}{\textit{Badanie EKG}}                                                                 \\*
    \cmidrule(r){2-4}
    50            & nowy załamek Q                                        & dychotomiczny        & 0 - brak, 1 - obecny      \\
    51            & jakikolwiek załamek~Q                                 & dychotomiczny        & 0 - brak, 1 - obecny      \\
    52            & nowe uniesienie odcinka~ST                            & dychotomiczny        & 0 - brak, 1 - obecny      \\
    53            & jakiekolwiek uniesienie odcinka~ST                    & dychotomiczny        & 0 - brak, 1 - obecny      \\
    54            & nowe obniżenie odcinka~ST                             & dychotomiczny        & 0 - brak, 1 - obecny      \\
    55            & jakiekolwiek obniżenie odcinka~ST                     & dychotomiczny        & 0 - brak, 1 - obecny      \\
    56            & nowy odwrócony załamek~T                              & dychotomiczny        & 0 - brak, 1 - obecny      \\
    57            & jakikolwiek odwrócony załamek~T                       & dychotomiczny        & 0 - brak, 1 - obecny      \\
    58            & nowe zaburzenie przewodnictwa śródkomorowego          & dychotomiczny        & 0 - brak, 1 - obecny      \\
    59            & jakiekolwiek zaburzenie przewodnictwa śródkomorowego  & dychotomiczny        & 0 - brak, 1 - obecny      \\
    \bottomrule
\end{longtabu}

\begin{table}[H]
    \caption{Opis wartości cechy \textit{miejsce~bólu}.}\label{tab:cecha_3}
    \begin{tabu}{rX[l]}
        \toprule
        \textbf{Wartość} & \textbf{Znaczenie}          \\
        \midrule

        1                & zamostkowy                  \\
        2                & lewa strona, w okol. serca  \\
        3                & prawa strona na wys. serca  \\
        4                & lewy bok klatki piersiowej  \\
        5                & prawy bok klatki piersiowej \\
        6                & brzuch                      \\
        7                & plecy                       \\
        8                & inne                        \\
        \bottomrule
    \end{tabu}
\end{table}

\begin{table}[H]
    \caption{Opis wartości cechy \textit{promieniowanie bólu w klatce piersiowej}.}\label{tab:cecha_4}
    \begin{tabu}{rX[l]}
        \toprule
        \textbf{Wartość} & \textbf{Znaczenie} \\
        \midrule

        1                & szyja              \\
        2                & szczęka            \\
        3                & lewe ramię         \\
        4                & lewa ręka          \\
        5                & prawe ramię        \\
        6                & plecy              \\
        7                & brzuch             \\
        8                & inne               \\
        \bottomrule
    \end{tabu}
\end{table}

\begin{table}[H]
    \caption{Opis wartości cechy \textit{charakter bólu}.}\label{tab:cecha_5}
    \begin{tabu}{rX[l]}
        \toprule
        \textbf{Wartość} & \textbf{Znaczenie} \\
        \midrule

        1                & szyja              \\
        2                & szczęka            \\
        3                & lewe ramię         \\
        4                & lewa ręka          \\
        5                & prawe ramię        \\
        6                & plecy              \\
        7                & brzuch             \\
        8                & inne               \\
        \bottomrule
    \end{tabu}
\end{table}

\begin{table}[H]
    \caption{Opis wartości cechy \textit{początek występowania bólu}.}\label{tab:cecha_6}
    \begin{tabu}{rX[l]}
        \toprule
        \textbf{Wartość} & \textbf{Znaczenie} \\
        \midrule
        1                & podczas wysiłku    \\
        2                & w spoczynku        \\
        3                & podczas snu        \\
        \bottomrule
    \end{tabu}
\end{table}

\begin{table}[H]
    \caption{Opis wartości cechy \textit{długość trwania ostatniego bólu}.}\label{tab:cecha_8}
    \begin{tabu}{rX[l]}
        \toprule
        \textbf{Wartość} & \textbf{Znaczenie} \\
        \midrule
        1                & poniżej 5 min      \\
        2                & 5 - 30 min         \\
        3                & 30 - 60 min        \\
        4                & 1 - 6 godz.        \\
        5                & 6 - 12 godz.       \\
        6                & powyżej 12 godz.   \\
        \bottomrule
    \end{tabu}
\end{table}

\begin{table}[H]
    \caption{Opis wartości cechy \textit{czynniki paliatywne}.}\label{tab:cecha_15}
    \begin{tabu}{rX[l]}
        \toprule
        \textbf{Wartość} & \textbf{Znaczenie}                 \\
        \midrule
        1                & brak                               \\
        2                & nitrogliceryna w ciągu 5 min       \\
        3                & nitrogliceryna po upływie 5 min    \\
        4                & leki zobojętniające kwas żołądkowy \\
        5                & znieczulenie poza morfiną          \\
        6                & morfina                            \\
        \bottomrule
    \end{tabu}
\end{table}



\section{Algorytm k-NN}
Algorytm k-Najbliższych Sąsiadów stanowi jedną z najbardziej prostych metod klasyfikacji. Jest łatwy w implementacji w swojej podstawowej formie oraz wykonuje dość złożone zadania klasyfikacyjne. k-NN to przykład klasyfikatora leniwego, czyli takiego, który wyciąga wnioski dopiero przy procedurze predykcji bazując na wiedzy o danych zebranej podczas procesu uczenia. W następnych etapach swojego działania omawiany algorytm wyszukuje k najbliższych wzorców ze zbioru uczącego, oblicza do nich odległość za pomocą metryki, jako predykcję zwraca tę klasę, która występuje częściej w obrębie lokalnego sąsiedztwa. Przyjmuje się, że k powinno być liczbą nieparzystą, żeby uniknąć remisu w przypadku problemów binarnych, niemniej jednak nie ma żadnej liczby, która byłaby najlepsza.

\begin{algorithm}[!ht]
    \label{algorytm}
    \SetKwInOut{Input}{Input}
    \Input{$X$ = zestaw uczący\\$L$ = etykiety klas zestawu \\ $x_{q}$ = niesklasyfikowana próbka \\ $k$ = liczba sąsiadów }
    \BlankLine
    \SetAlgoVlined
    \For{$(x', l') \in X$}{
        Oblicz odległość d($x'$, $x_{q}$)}

    Posortuj rosnąco obliczone odległości elementów zestawu uczącego $X$ od $x_{q}$ \\
    Policz wystąpienia każdej z klas w $L$ pośród najbliższych $k$ sąsiadów $x_{q}$ \\
    Przydziel $x_{q}$ do najczęściej występującej klasy
    \caption{K Nearest Neighbors}
\end{algorithm}

\subsection{Miary odległości}
Istotnym elementem algorytmu k-NN jest odległość, na podstawie której wyznacza się najbliższych sąsiadów. Wybrano dwie  metryki, które zostaną wykorzystane w projekcie.
Pierwsza metryka to odległość euklidesowa. Stanowi jedną na najczęściej wykorzystywanych metryk, za jej pomocą można obliczyć odległość między dwoma punktami (x, y) na płaszczyźnie (wzór \ref{eq:e}).

\begin{center}
    \begin{equation}
        \label{eq:e}
        d_{e}\left( x,y\right)   = \sqrt {\sum _{i=1}^{n}  \left( x_{i}-y_{i}\right)^2 }
    \end{equation}
\end{center}
\noindent
Drugą metryką jest Manhattan, którą oblicza się stosując wzór \ref{eq:m}.

\begin{center}
    \begin{equation}
        \label{eq:m}
        d_{m}\left( x,y\right)   = \sum_{i=1}^n |x_i-y_i|
    \end{equation}
\end{center}

\noindent
Trzecią metryką jest metryka Czebyszewa, którą oblicza się stosując wzór \ref{eq:ch}.

\begin{center}
    \begin{equation}
        \label{eq:ch}
        d_{ch}\left( x,y\right)   = \max_{i} |x_i-y_i|
    \end{equation}
\end{center}


\subsection{Implementacja środowiska eksperymentowania}
Do zaimplementowania środowiska eksperymentowania wykorzystano język Python, ponieważ wykorzystywaną biblioteką do uczenia maszynowego jest scikit-learn\cite{scikit}.

\section{Wyniki ewaluacji eksperymentalnej}
Walidacja została dokonana z użyciem 5 razy powtórzonej 2-krotnej walidacji krzyżowej, a jakość klasyfikacji mierzona metryką dokładności (\textit{accuracy}).
Wyniki pokazane zostały na rysunkach \ref{fig:euclidean}, \ref{fig:manhattan} i \ref{fig:chebyshev} dla każdej z metryk odległości.

\subsection{Wnioski}
Wyniki eksperymentów uwidaczniają jak zmienia się dokładność klasyfikacji na podstawie liczby cech, liczby sąsiadów branych pod uwagę w tym procesie oraz metryk odległości.

Dla metryki euklidesowej maksymalna dokładność 65,68\% osiągana jest dla 5 sąsiadów i 34 cech. Dokładność ta wzrasta do poziomu 60-65\% dla liczby cech od 13 do 35, a potem maleje schodkowo.

Dla metryki Manhattan maksymalna dokładność 72,16\% osiągana jest dla 8 sąsiadów i 47 cech. Dokładność ta wzrasta do poziomu ok. 70\% dla liczby cech od 13 i utrzymuje się na stałym poziomie niezależnie od dalszego przyrostu liczby cech.

Dla metryki Czebyszewa maksymalna dokładność 59,08\% osiągana jest dla 3 sąsiadów i 16 cech. Dokładność ta wzrasta do poziomu 60-65\% dla liczby cech od 13 do 35, a potem maleje schodkowo. Z uwagi na charakterystykę tej metryki dokładność nie podlega żadnym wahaniom w określonych przedziałach. Z uwagi na charakterystykę zestawu danych uczących dla liczby cech od 6 do 8 występuje spadek dokładności.

Metryką pozwalającą osiągnąć największą dokładność klasyfikacji jest metryka Manhattan.

\begin{landscape}

    \begin{figure}[h]
        \includegraphics[width=\linewidth]{./img/plot_euclidean.png}
        \caption{Wyniki ewaluacji eksperymentalnej dla metryki euklidesowej} \label{fig:euclidean}
    \end{figure}

    \begin{figure}[h]
        \includegraphics[width=\linewidth]{./img/plot_manhattan.png}
        \caption{Wyniki ewaluacji eksperymentalnej dla metryki Manhattan} \label{fig:manhattan}
    \end{figure}

    \begin{figure}[h]
        \includegraphics[width=\linewidth]{./img/plot_chebyshev.png}
        \caption{Wyniki ewaluacji eksperymentalnej dla metryki Czebyszewa} \label{fig:chebyshev}
    \end{figure}

\end{landscape}

\clearpage
%
% ---- Bibliography ----
%
% BibTeX users should specify bibliography style 'splncs04'.
% References will then be sorted and formatted in the correct style.
%
\bibliographystyle{../src/llncs2e/splncs04}
\bibliography{bibliography}
%

\end{document}
